% Options for packages loaded elsewhere
% Options for packages loaded elsewhere
\PassOptionsToPackage{unicode}{hyperref}
\PassOptionsToPackage{hyphens}{url}
\PassOptionsToPackage{dvipsnames,svgnames,x11names}{xcolor}
%
\documentclass[
  letterpaper,
  DIV=11,
  numbers=noendperiod]{scrreprt}
\usepackage{xcolor}
\usepackage{amsmath,amssymb}
\setcounter{secnumdepth}{5}
\usepackage{iftex}
\ifPDFTeX
  \usepackage[T1]{fontenc}
  \usepackage[utf8]{inputenc}
  \usepackage{textcomp} % provide euro and other symbols
\else % if luatex or xetex
  \usepackage{unicode-math} % this also loads fontspec
  \defaultfontfeatures{Scale=MatchLowercase}
  \defaultfontfeatures[\rmfamily]{Ligatures=TeX,Scale=1}
\fi
\usepackage{lmodern}
\ifPDFTeX\else
  % xetex/luatex font selection
\fi
% Use upquote if available, for straight quotes in verbatim environments
\IfFileExists{upquote.sty}{\usepackage{upquote}}{}
\IfFileExists{microtype.sty}{% use microtype if available
  \usepackage[]{microtype}
  \UseMicrotypeSet[protrusion]{basicmath} % disable protrusion for tt fonts
}{}
\makeatletter
\@ifundefined{KOMAClassName}{% if non-KOMA class
  \IfFileExists{parskip.sty}{%
    \usepackage{parskip}
  }{% else
    \setlength{\parindent}{0pt}
    \setlength{\parskip}{6pt plus 2pt minus 1pt}}
}{% if KOMA class
  \KOMAoptions{parskip=half}}
\makeatother
% Make \paragraph and \subparagraph free-standing
\makeatletter
\ifx\paragraph\undefined\else
  \let\oldparagraph\paragraph
  \renewcommand{\paragraph}{
    \@ifstar
      \xxxParagraphStar
      \xxxParagraphNoStar
  }
  \newcommand{\xxxParagraphStar}[1]{\oldparagraph*{#1}\mbox{}}
  \newcommand{\xxxParagraphNoStar}[1]{\oldparagraph{#1}\mbox{}}
\fi
\ifx\subparagraph\undefined\else
  \let\oldsubparagraph\subparagraph
  \renewcommand{\subparagraph}{
    \@ifstar
      \xxxSubParagraphStar
      \xxxSubParagraphNoStar
  }
  \newcommand{\xxxSubParagraphStar}[1]{\oldsubparagraph*{#1}\mbox{}}
  \newcommand{\xxxSubParagraphNoStar}[1]{\oldsubparagraph{#1}\mbox{}}
\fi
\makeatother

\usepackage{color}
\usepackage{fancyvrb}
\newcommand{\VerbBar}{|}
\newcommand{\VERB}{\Verb[commandchars=\\\{\}]}
\DefineVerbatimEnvironment{Highlighting}{Verbatim}{commandchars=\\\{\}}
% Add ',fontsize=\small' for more characters per line
\usepackage{framed}
\definecolor{shadecolor}{RGB}{241,243,245}
\newenvironment{Shaded}{\begin{snugshade}}{\end{snugshade}}
\newcommand{\AlertTok}[1]{\textcolor[rgb]{0.68,0.00,0.00}{#1}}
\newcommand{\AnnotationTok}[1]{\textcolor[rgb]{0.37,0.37,0.37}{#1}}
\newcommand{\AttributeTok}[1]{\textcolor[rgb]{0.40,0.45,0.13}{#1}}
\newcommand{\BaseNTok}[1]{\textcolor[rgb]{0.68,0.00,0.00}{#1}}
\newcommand{\BuiltInTok}[1]{\textcolor[rgb]{0.00,0.23,0.31}{#1}}
\newcommand{\CharTok}[1]{\textcolor[rgb]{0.13,0.47,0.30}{#1}}
\newcommand{\CommentTok}[1]{\textcolor[rgb]{0.37,0.37,0.37}{#1}}
\newcommand{\CommentVarTok}[1]{\textcolor[rgb]{0.37,0.37,0.37}{\textit{#1}}}
\newcommand{\ConstantTok}[1]{\textcolor[rgb]{0.56,0.35,0.01}{#1}}
\newcommand{\ControlFlowTok}[1]{\textcolor[rgb]{0.00,0.23,0.31}{\textbf{#1}}}
\newcommand{\DataTypeTok}[1]{\textcolor[rgb]{0.68,0.00,0.00}{#1}}
\newcommand{\DecValTok}[1]{\textcolor[rgb]{0.68,0.00,0.00}{#1}}
\newcommand{\DocumentationTok}[1]{\textcolor[rgb]{0.37,0.37,0.37}{\textit{#1}}}
\newcommand{\ErrorTok}[1]{\textcolor[rgb]{0.68,0.00,0.00}{#1}}
\newcommand{\ExtensionTok}[1]{\textcolor[rgb]{0.00,0.23,0.31}{#1}}
\newcommand{\FloatTok}[1]{\textcolor[rgb]{0.68,0.00,0.00}{#1}}
\newcommand{\FunctionTok}[1]{\textcolor[rgb]{0.28,0.35,0.67}{#1}}
\newcommand{\ImportTok}[1]{\textcolor[rgb]{0.00,0.46,0.62}{#1}}
\newcommand{\InformationTok}[1]{\textcolor[rgb]{0.37,0.37,0.37}{#1}}
\newcommand{\KeywordTok}[1]{\textcolor[rgb]{0.00,0.23,0.31}{\textbf{#1}}}
\newcommand{\NormalTok}[1]{\textcolor[rgb]{0.00,0.23,0.31}{#1}}
\newcommand{\OperatorTok}[1]{\textcolor[rgb]{0.37,0.37,0.37}{#1}}
\newcommand{\OtherTok}[1]{\textcolor[rgb]{0.00,0.23,0.31}{#1}}
\newcommand{\PreprocessorTok}[1]{\textcolor[rgb]{0.68,0.00,0.00}{#1}}
\newcommand{\RegionMarkerTok}[1]{\textcolor[rgb]{0.00,0.23,0.31}{#1}}
\newcommand{\SpecialCharTok}[1]{\textcolor[rgb]{0.37,0.37,0.37}{#1}}
\newcommand{\SpecialStringTok}[1]{\textcolor[rgb]{0.13,0.47,0.30}{#1}}
\newcommand{\StringTok}[1]{\textcolor[rgb]{0.13,0.47,0.30}{#1}}
\newcommand{\VariableTok}[1]{\textcolor[rgb]{0.07,0.07,0.07}{#1}}
\newcommand{\VerbatimStringTok}[1]{\textcolor[rgb]{0.13,0.47,0.30}{#1}}
\newcommand{\WarningTok}[1]{\textcolor[rgb]{0.37,0.37,0.37}{\textit{#1}}}

\usepackage{longtable,booktabs,array}
\usepackage{calc} % for calculating minipage widths
% Correct order of tables after \paragraph or \subparagraph
\usepackage{etoolbox}
\makeatletter
\patchcmd\longtable{\par}{\if@noskipsec\mbox{}\fi\par}{}{}
\makeatother
% Allow footnotes in longtable head/foot
\IfFileExists{footnotehyper.sty}{\usepackage{footnotehyper}}{\usepackage{footnote}}
\makesavenoteenv{longtable}
\usepackage{graphicx}
\makeatletter
\newsavebox\pandoc@box
\newcommand*\pandocbounded[1]{% scales image to fit in text height/width
  \sbox\pandoc@box{#1}%
  \Gscale@div\@tempa{\textheight}{\dimexpr\ht\pandoc@box+\dp\pandoc@box\relax}%
  \Gscale@div\@tempb{\linewidth}{\wd\pandoc@box}%
  \ifdim\@tempb\p@<\@tempa\p@\let\@tempa\@tempb\fi% select the smaller of both
  \ifdim\@tempa\p@<\p@\scalebox{\@tempa}{\usebox\pandoc@box}%
  \else\usebox{\pandoc@box}%
  \fi%
}
% Set default figure placement to htbp
\def\fps@figure{htbp}
\makeatother


% definitions for citeproc citations
\NewDocumentCommand\citeproctext{}{}
\NewDocumentCommand\citeproc{mm}{%
  \begingroup\def\citeproctext{#2}\cite{#1}\endgroup}
\makeatletter
 % allow citations to break across lines
 \let\@cite@ofmt\@firstofone
 % avoid brackets around text for \cite:
 \def\@biblabel#1{}
 \def\@cite#1#2{{#1\if@tempswa , #2\fi}}
\makeatother
\newlength{\cslhangindent}
\setlength{\cslhangindent}{1.5em}
\newlength{\csllabelwidth}
\setlength{\csllabelwidth}{3em}
\newenvironment{CSLReferences}[2] % #1 hanging-indent, #2 entry-spacing
 {\begin{list}{}{%
  \setlength{\itemindent}{0pt}
  \setlength{\leftmargin}{0pt}
  \setlength{\parsep}{0pt}
  % turn on hanging indent if param 1 is 1
  \ifodd #1
   \setlength{\leftmargin}{\cslhangindent}
   \setlength{\itemindent}{-1\cslhangindent}
  \fi
  % set entry spacing
  \setlength{\itemsep}{#2\baselineskip}}}
 {\end{list}}
\usepackage{calc}
\newcommand{\CSLBlock}[1]{\hfill\break\parbox[t]{\linewidth}{\strut\ignorespaces#1\strut}}
\newcommand{\CSLLeftMargin}[1]{\parbox[t]{\csllabelwidth}{\strut#1\strut}}
\newcommand{\CSLRightInline}[1]{\parbox[t]{\linewidth - \csllabelwidth}{\strut#1\strut}}
\newcommand{\CSLIndent}[1]{\hspace{\cslhangindent}#1}



\setlength{\emergencystretch}{3em} % prevent overfull lines

\providecommand{\tightlist}{%
  \setlength{\itemsep}{0pt}\setlength{\parskip}{0pt}}



 


\KOMAoption{captions}{tableheading}
\makeatletter
\@ifpackageloaded{tcolorbox}{}{\usepackage[skins,breakable]{tcolorbox}}
\@ifpackageloaded{fontawesome5}{}{\usepackage{fontawesome5}}
\definecolor{quarto-callout-color}{HTML}{909090}
\definecolor{quarto-callout-note-color}{HTML}{0758E5}
\definecolor{quarto-callout-important-color}{HTML}{CC1914}
\definecolor{quarto-callout-warning-color}{HTML}{EB9113}
\definecolor{quarto-callout-tip-color}{HTML}{00A047}
\definecolor{quarto-callout-caution-color}{HTML}{FC5300}
\definecolor{quarto-callout-color-frame}{HTML}{acacac}
\definecolor{quarto-callout-note-color-frame}{HTML}{4582ec}
\definecolor{quarto-callout-important-color-frame}{HTML}{d9534f}
\definecolor{quarto-callout-warning-color-frame}{HTML}{f0ad4e}
\definecolor{quarto-callout-tip-color-frame}{HTML}{02b875}
\definecolor{quarto-callout-caution-color-frame}{HTML}{fd7e14}
\makeatother
\makeatletter
\@ifpackageloaded{bookmark}{}{\usepackage{bookmark}}
\makeatother
\makeatletter
\@ifpackageloaded{caption}{}{\usepackage{caption}}
\AtBeginDocument{%
\ifdefined\contentsname
  \renewcommand*\contentsname{Table of contents}
\else
  \newcommand\contentsname{Table of contents}
\fi
\ifdefined\listfigurename
  \renewcommand*\listfigurename{List of Figures}
\else
  \newcommand\listfigurename{List of Figures}
\fi
\ifdefined\listtablename
  \renewcommand*\listtablename{List of Tables}
\else
  \newcommand\listtablename{List of Tables}
\fi
\ifdefined\figurename
  \renewcommand*\figurename{Figure}
\else
  \newcommand\figurename{Figure}
\fi
\ifdefined\tablename
  \renewcommand*\tablename{Table}
\else
  \newcommand\tablename{Table}
\fi
}
\@ifpackageloaded{float}{}{\usepackage{float}}
\floatstyle{ruled}
\@ifundefined{c@chapter}{\newfloat{codelisting}{h}{lop}}{\newfloat{codelisting}{h}{lop}[chapter]}
\floatname{codelisting}{Listing}
\newcommand*\listoflistings{\listof{codelisting}{List of Listings}}
\makeatother
\makeatletter
\makeatother
\makeatletter
\@ifpackageloaded{caption}{}{\usepackage{caption}}
\@ifpackageloaded{subcaption}{}{\usepackage{subcaption}}
\makeatother
\usepackage{bookmark}
\IfFileExists{xurl.sty}{\usepackage{xurl}}{} % add URL line breaks if available
\urlstyle{same}
\hypersetup{
  pdftitle={Multilevel Workshop},
  pdfauthor={Andrew Grogan-Kaylor},
  colorlinks=true,
  linkcolor={blue},
  filecolor={Maroon},
  citecolor={Blue},
  urlcolor={Blue},
  pdfcreator={LaTeX via pandoc}}


\title{Multilevel Workshop}
\author{Andrew Grogan-Kaylor}
\date{2026-01-12}
\begin{document}
\maketitle

\renewcommand*\contentsname{Table of contents}
{
\hypersetup{linkcolor=}
\setcounter{tocdepth}{2}
\tableofcontents
}
\listoffigures
\listoftables

\bookmarksetup{startatroot}

\chapter{Introduction}\label{introduction}

\begin{quote}
``Listening to the world. Well, I did that, and I still do it. I still
do it.'' (Mary Oliver in Oliver and Tippett 2015)
\end{quote}

This site contains materials for a workshop on multilevel modeling.

\section{Background}\label{background}

Multilevel models are useful when you have data that are nested or
clustered inside social units such as schools, neighborhoods, states, or
countries.

Multilevel models are also useful when you have longitudinal data where
repeated measures are collected for study participants.

\begin{tcolorbox}[enhanced jigsaw, rightrule=.15mm, leftrule=.75mm, opacitybacktitle=0.6, breakable, colframe=quarto-callout-tip-color-frame, coltitle=black, opacityback=0, toprule=.15mm, title=\textcolor{quarto-callout-tip-color}{\faLightbulb}\hspace{0.5em}{The Importance of Multilevel Models}, toptitle=1mm, colbacktitle=quarto-callout-tip-color!10!white, bottomrule=.15mm, left=2mm, colback=white, bottomtitle=1mm, titlerule=0mm, arc=.35mm]

Multilevel models may improve one's statistical inferences in two
important substantive ways.

\begin{itemize}
\tightlist
\item
  Multilevel models adjust standard errors for clustering, and thus
  calculate appropriate p values. Failure to use a model that accounts
  for the clustering in the data may lead to improperly calculated
  standard errors and p values, possibly leading to false attributions
  of statistical significance (false positives) (see
  Chapter~\ref{sec-clustering}).
\item
  Multilevel models adjust regression coefficients (\(\beta\)'s) for the
  presence of clustering. Failure to use a model that accounts for the
  clustering in the data may lead to improperly calculated regression
  coefficients (\(\beta\)'s) which may have the wrong magnitude, the
  wrong statistical significance, and even the wrong sign (see
  Chapter~\ref{sec-multilevelstructure}).
\end{itemize}

\end{tcolorbox}

\section{Simulated Multilevel Data}\label{simulated-multilevel-data}

The data used in these workshop materials are \emph{simulated} data on
parents, children and families. The data are simulated to come from 30
hypothetical countries around the world. These are the same data used
and discussed in my book
\emph{\href{https://academic.oup.com/book/60530}{Multilevel Thinking:
Discovering Variation, Universals, and Particulars in Cross-Cultural
Research}}.

There are two versions of the data: a \emph{cross-sectional} data set
from a single point in time; a \emph{longitudinal} version of the data
spanning several time points.

\begin{tcolorbox}[enhanced jigsaw, rightrule=.15mm, leftrule=.75mm, opacitybacktitle=0.6, breakable, colframe=quarto-callout-note-color-frame, coltitle=black, opacityback=0, toprule=.15mm, title=\textcolor{quarto-callout-note-color}{\faInfo}\hspace{0.5em}{The Data Can Be Downloaded Here:}, toptitle=1mm, colbacktitle=quarto-callout-note-color!10!white, bottomrule=.15mm, left=2mm, colback=white, bottomtitle=1mm, titlerule=0mm, arc=.35mm]

\begin{itemize}
\tightlist
\item
  \href{https://github.com/agrogan1/multilevel-workshop/raw/refs/heads/main/simulated_multilevel_data.dta}{Cross
  Sectional Data}
\item
  \href{https://github.com/agrogan1/multilevel-workshop/raw/refs/heads/main/simulated_multilevel_longitudinal_data.dta}{Longitudinal
  Data}
\end{itemize}

\end{tcolorbox}

\begin{longtable}[]{@{}
  >{\centering\arraybackslash}p{(\linewidth - 4\tabcolsep) * \real{0.0833}}
  >{\centering\arraybackslash}p{(\linewidth - 4\tabcolsep) * \real{0.3056}}
  >{\centering\arraybackslash}p{(\linewidth - 4\tabcolsep) * \real{0.4306}}@{}}

\caption{\label{tbl-describe}Variables in Simulated Multilevel Data}

\tabularnewline

\toprule\noalign{}
\begin{minipage}[b]{\linewidth}\centering
pos
\end{minipage} & \begin{minipage}[b]{\linewidth}\centering
variable
\end{minipage} & \begin{minipage}[b]{\linewidth}\centering
label
\end{minipage} \\
\midrule\noalign{}
\endhead
\bottomrule\noalign{}
\endlastfoot
1 & country & country id \\
2 & HDI & Human Development Index \\
3 & family & family id \\
4 & id & unique country family id \\
5 & identity & hypothetical identity group variable \\
6 & intervention & recieved intervention \\
7 & physical\_punishment & physical punishment in past week \\
8 & warmth & parental warmth in past week \\
9 & outcome & beneficial outcome \\

\end{longtable}

\begin{longtable}[]{@{}
  >{\centering\arraybackslash}p{(\linewidth - 16\tabcolsep) * \real{0.1000}}
  >{\centering\arraybackslash}p{(\linewidth - 16\tabcolsep) * \real{0.0600}}
  >{\centering\arraybackslash}p{(\linewidth - 16\tabcolsep) * \real{0.0900}}
  >{\centering\arraybackslash}p{(\linewidth - 16\tabcolsep) * \real{0.0800}}
  >{\centering\arraybackslash}p{(\linewidth - 16\tabcolsep) * \real{0.1100}}
  >{\centering\arraybackslash}p{(\linewidth - 16\tabcolsep) * \real{0.1500}}
  >{\centering\arraybackslash}p{(\linewidth - 16\tabcolsep) * \real{0.2200}}
  >{\centering\arraybackslash}p{(\linewidth - 16\tabcolsep) * \real{0.0900}}
  >{\centering\arraybackslash}p{(\linewidth - 16\tabcolsep) * \real{0.1000}}@{}}

\caption{\label{tbl-head}Sample Data From Simulated Multilevel Data}

\tabularnewline

\toprule\noalign{}
\begin{minipage}[b]{\linewidth}\centering
country
\end{minipage} & \begin{minipage}[b]{\linewidth}\centering
HDI
\end{minipage} & \begin{minipage}[b]{\linewidth}\centering
family
\end{minipage} & \begin{minipage}[b]{\linewidth}\centering
id
\end{minipage} & \begin{minipage}[b]{\linewidth}\centering
identity
\end{minipage} & \begin{minipage}[b]{\linewidth}\centering
intervention
\end{minipage} & \begin{minipage}[b]{\linewidth}\centering
physical\_punishment
\end{minipage} & \begin{minipage}[b]{\linewidth}\centering
warmth
\end{minipage} & \begin{minipage}[b]{\linewidth}\centering
outcome
\end{minipage} \\
\midrule\noalign{}
\endhead
\bottomrule\noalign{}
\endlastfoot
15 & 77 & 59 & 15.59 & 0 & 0 & 3 & 6 & 57.13 \\
2 & 83 & 49 & 2.49 & 0 & 1 & 3 & 4 & 57.51 \\
16 & 57 & 50 & 16.50 & 1 & 0 & 1 & 5 & 51 \\
24 & 82 & 38 & 24.38 & 0 & 0 & 1 & 6 & 59.41 \\
28 & 53 & 32 & 28.32 & 0 & 0 & 5 & 2 & 45.1 \\

\end{longtable}

\bookmarksetup{startatroot}

\chapter{The Importance of Accounting for Clustered
Data}\label{sec-clustering}

\section{Grouped and Individual Data}\label{grouped-and-individual-data}

Bland and Altman (1994) suggested the following procedure for simulating
some data:

\begin{quote}
``The data were generated from random numbers, and there is no relation
between X and Y at all. Firstly, values of X and Y were generated for
each `subject,' then a further random number was added to make the
individual observation.'' (Bland and Altman 1994)
\end{quote}

So\ldots{} we follow their procedure.

\begin{tcolorbox}[enhanced jigsaw, rightrule=.15mm, leftrule=.75mm, opacitybacktitle=0.6, breakable, colframe=quarto-callout-note-color-frame, coltitle=black, opacityback=0, toprule=.15mm, title=\textcolor{quarto-callout-note-color}{\faInfo}\hspace{0.5em}{Simulating The Data}, toptitle=1mm, colbacktitle=quarto-callout-note-color!10!white, bottomrule=.15mm, left=2mm, colback=white, bottomtitle=1mm, titlerule=0mm, arc=.35mm]

The graph below illustrates the process of simulating the data.

\end{tcolorbox}

\pandocbounded{\includegraphics[keepaspectratio]{clustering_files/figure-pdf/unnamed-chunk-3-1.pdf}}

\section{Analyses}\label{analyses}

\subsection{OLS}\label{ols}

An OLS analysis indicates that there is a statistically significant
association of \(x\) and \(y\).

\begin{verbatim}
                         OLS1  
-------------------------------
x_individual           1.046 **
Intercept              4.488   
Number of observations    25   
-------------------------------
** p<.01, * p<.05
\end{verbatim}

\subsection{MLM}\label{mlm}

In contrast, an MLM analysis (correctly) finds that there is no
statistically significant association of \(x\) and \(y\).

\begin{verbatim}
                          MLM1  
--------------------------------
x_individual            0.039   
Intercept              97.005 **
var(_cons)             74.523   
var(e)                  0.594   
Number of observations     25   
--------------------------------
** p<.01, * p<.05
\end{verbatim}

\subsection{Compare OLS and MLM}\label{compare-ols-and-mlm}

\begin{verbatim}
                         OLS1      MLM1  
-----------------------------------------
x_individual           1.046 **  0.039   
Intercept              4.488    97.005 **
var(_cons)                      74.523   
var(e)                           0.594   
Number of observations    25        25   
-----------------------------------------
** p<.01, * p<.05
\end{verbatim}

\bookmarksetup{startatroot}

\chapter{\texorpdfstring{Estimation of \(\beta\)
Coefficients}{Estimation of \textbackslash beta Coefficients}}\label{sec-multilevelstructure}

\section{Introduction}\label{introduction-1}

Associations between two variables can be \emph{very different} (or even
\emph{reversed}) depending upon whether or not the analysis is ``aware''
of the grouped, nested, or clustered nature of the data (Nieuwenhuis
2015; Diez Roux 2003; Gelman et al. 2007). In multilevel analysis, the
groups are often schools, neighborhoods, communities, or countries.

A model that is ``aware'' of the clustered nature of the data may
provide very different--likely better--substantive conclusions than a
model that is not aware of the clustered nature of the data. This
phenomena is closely related to the ``ecological fallacy'': the idea
that group level and individual level relationships are not necessarily
the same (Firebaugh 2001).

\section{Graphs}\label{graphs}

\subsection{A ``Naive'' Graph}\label{a-naive-graph}

This ``naive'' graph is unaware of the grouped nature of the data.

\pandocbounded{\includegraphics[keepaspectratio]{multilevel-structure_files/figure-pdf/unnamed-chunk-4-1.pdf}}

\subsection{An ``Aware'' Graph}\label{an-aware-graph}

This ``aware'' graph is aware of the grouped nature of the data.

\pandocbounded{\includegraphics[keepaspectratio]{multilevel-structure_files/figure-pdf/unnamed-chunk-5-1.pdf}}

\section{Regressions}\label{regressions}

\subsection{A ``Naive'' OLS Analysis}\label{a-naive-ols-analysis}

The OLS model with only \emph{x} as a covariate is not aware of the
grouped structure of the data, and the coefficient for \emph{x} reflects
this.

\begin{verbatim}
                          OLS2  
--------------------------------
x                      -0.761 **
Intercept              57.057 **
Number of observations     30   
--------------------------------
** p<.01, * p<.05
\end{verbatim}

\subsection{An ``Aware'' MLM Analysis}\label{an-aware-mlm-analysis}

The multilevel model is aware of the grouped structure of the data, and
the coefficient for \emph{x} reflects this.

\begin{verbatim}
                          MLM2   
---------------------------------
x                        1.166 **
Intercept               27.192 **
var(_cons)             312.623   
var(e)                   0.806   
Number of observations      30   
---------------------------------
** p<.01, * p<.05
\end{verbatim}

\subsection{Compare The Models}\label{compare-the-models}

\begin{verbatim}
                          OLS2      MLM2   
-------------------------------------------
x                      -0.761 **   1.166 **
Intercept              57.057 **  27.192 **
var(_cons)                       312.623   
var(e)                             0.806   
Number of observations     30         30   
-------------------------------------------
** p<.01, * p<.05
\end{verbatim}

\section{A Thought Experiment}\label{a-thought-experiment}

When might a situation like this arise in practice? This is surprisingly
difficult to think through.

Imagine that \emph{x} is a protective factor, or an intervention or
treatment. Imagine that \emph{y} is a desirable outcome, like improved
mental health or psychological well being.

Now imagine that people provide more of the protective factor or more of
the intervention in communities where there are lower levels of the
desirable outcome. If we think about it, this is a very plausible
situation.

\begin{tcolorbox}[enhanced jigsaw, rightrule=.15mm, leftrule=.75mm, opacitybacktitle=0.6, breakable, colframe=quarto-callout-tip-color-frame, coltitle=black, opacityback=0, toprule=.15mm, title=\textcolor{quarto-callout-tip-color}{\faLightbulb}\hspace{0.5em}{A Naive Analysis Would Misconstrue The Results}, toptitle=1mm, colbacktitle=quarto-callout-tip-color!10!white, bottomrule=.15mm, left=2mm, colback=white, bottomtitle=1mm, titlerule=0mm, arc=.35mm]

A naive analysis that was unaware of the grouped nature of the data
would therefore misconstrue the results, suggesting that the
intervention was harmful, when it was in fact helpful.

\end{tcolorbox}

\pandocbounded{\includegraphics[keepaspectratio]{multilevel-structure_files/figure-pdf/unnamed-chunk-9-1.pdf}}

These data are constructed to provide this kind of extreme example, but
it easy to see how multilevel analysis may provide better answers than
we would get if we ignored the grouped nature of the data.

\bookmarksetup{startatroot}

\chapter{Two Level Cross Sectional; And Three Level Longitudinal
Models}\label{two-level-cross-sectional-and-three-level-longitudinal-models}

\section{Cross Sectional Model}\label{cross-sectional-model}

\subsection{Get Data}\label{get-data}

\begin{Shaded}
\begin{Highlighting}[]

\KeywordTok{use} \StringTok{"simulated\_multilevel\_data.dta"}\NormalTok{, }\KeywordTok{clear}
\end{Highlighting}
\end{Shaded}

\subsection{The Equation}\label{the-equation}

\[\begin{aligned}
\text{outcome}_{ij} = \beta_0 + \beta_1 \text{parental warmth} + \beta_2 \text{physical punishment} + \\ \beta_3 \text{identity} + \beta_4 \text{intervention} + \beta_5 HDI + \\ u_{0j} + u_{1j} \times \text{parental warmth} + e_{ij} 
\end{aligned}\]

\subsection{Descriptive Statistics}\label{descriptive-statistics}

\begin{Shaded}
\begin{Highlighting}[]

\KeywordTok{summarize} \CommentTok{// descriptive statistics}
\end{Highlighting}
\end{Shaded}

\begin{verbatim}
    Variable |        Obs        Mean    Std. dev.       Min        Max
-------------+---------------------------------------------------------
     country |      3,000        15.5    8.656884          1         30
         HDI |      3,000    64.76667    17.24562         33         87
      family |      3,000        50.5    28.87088          1        100
          id |          0
    identity |      3,000    .4976667    .5000779          0          1
-------------+---------------------------------------------------------
intervention |      3,000    .4843333    .4998378          0          1
physical_p~t |      3,000    2.478667    1.360942          0          5
      warmth |      3,000    3.521667    1.888399          0          7
     outcome |      3,000    52.43327    6.530996   29.60798   74.83553
\end{verbatim}

\subsection{Spaghetti Plot}\label{spaghetti-plot}

\begin{Shaded}
\begin{Highlighting}[]
\NormalTok{spagplot outcome warmth, id(country) }\DecValTok{scheme}\NormalTok{(stcolor)}

\KeywordTok{graph} \KeywordTok{export}\NormalTok{ spagplot1.png, }\KeywordTok{width}\NormalTok{(1000) }\KeywordTok{replace}
\end{Highlighting}
\end{Shaded}

\begin{figure}[H]

{\centering \includegraphics[width=0.5\linewidth,height=\textheight,keepaspectratio]{spagplot1.png}

}

\caption{Spaghetti Plot of Outcome by Warmth by Country}

\end{figure}%

\subsection{Unconditional Model}\label{unconditional-model}

\subsubsection{Model}\label{model}

\begin{Shaded}
\begin{Highlighting}[]

\NormalTok{mixed outcome || country: }\CommentTok{// unconditional model}
\end{Highlighting}
\end{Shaded}

\begin{verbatim}
Performing EM optimization ...

Performing gradient-based optimization: 
Iteration 0:  Log likelihood = -9802.8371  
Iteration 1:  Log likelihood = -9802.8371  

Computing standard errors ...

Mixed-effects ML regression                           Number of obs    = 3,000
Group variable: country                               Number of groups =    30
                                                      Obs per group:
                                                                   min =   100
                                                                   avg = 100.0
                                                                   max =   100
                                                      Wald chi2(0)     =     .
Log likelihood = -9802.8371                           Prob > chi2      =     .

------------------------------------------------------------------------------
     outcome | Coefficient  Std. err.      z    P>|z|     [95% conf. interval]
-------------+----------------------------------------------------------------
       _cons |   52.43327   .3451217   151.93   0.000     51.75685     53.1097
------------------------------------------------------------------------------

------------------------------------------------------------------------------
  Random-effects parameters  |   Estimate   Std. err.     [95% conf. interval]
-----------------------------+------------------------------------------------
country: Identity            |
                  var(_cons) |   3.178658   .9226736      1.799552    5.614658
-----------------------------+------------------------------------------------
               var(Residual) |   39.46106   1.024013      37.50421       41.52
------------------------------------------------------------------------------
LR test vs. linear model: chibar2(01) = 166.31        Prob >= chibar2 = 0.0000
\end{verbatim}

\subsubsection{ICC}\label{icc}

\begin{Shaded}
\begin{Highlighting}[]

\KeywordTok{estat}\NormalTok{ icc}
\end{Highlighting}
\end{Shaded}

\begin{verbatim}
Intraclass correlation

------------------------------------------------------------------------------
                       Level |        ICC   Std. err.     [95% conf. interval]
-----------------------------+------------------------------------------------
                     country |   .0745469   .0201254      .0434963    .1248696
------------------------------------------------------------------------------
\end{verbatim}

\subsection{Conditional Model}\label{conditional-model}

\begin{Shaded}
\begin{Highlighting}[]

\NormalTok{mixed outcome warmth physical\_punishment }\KeywordTok{identity}\NormalTok{ i.intervention HDI || country: warmth }\CommentTok{// multilevel model}

\KeywordTok{est} \KeywordTok{store}\NormalTok{ crosssectional }\CommentTok{// store estimates}
\end{Highlighting}
\end{Shaded}

\begin{verbatim}
Performing EM optimization ...

Performing gradient-based optimization: 
Iteration 0:  Log likelihood = -9626.6279  
Iteration 1:  Log likelihood =  -9626.607  
Iteration 2:  Log likelihood =  -9626.607  

Computing standard errors ...

Mixed-effects ML regression                          Number of obs    =  3,000
Group variable: country                              Number of groups =     30
                                                     Obs per group:
                                                                  min =    100
                                                                  avg =  100.0
                                                                  max =    100
                                                     Wald chi2(5)     = 334.14
Log likelihood =  -9626.607                          Prob > chi2      = 0.0000

------------------------------------------------------------------------------------
           outcome | Coefficient  Std. err.      z    P>|z|     [95% conf. interval]
-------------------+----------------------------------------------------------------
            warmth |   .8345368   .0637213    13.10   0.000     .7096453    .9594282
physical_punishm~t |  -.9916657   .0797906   -12.43   0.000    -1.148052   -.8352791
          identity |  -.3004767   .2170295    -1.38   0.166    -.7258466    .1248933
    1.intervention |   .6396427   .2174519     2.94   0.003     .2134448    1.065841
               HDI |   -.003228   .0199257    -0.16   0.871    -.0422817    .0358256
             _cons |   51.99991   1.371257    37.92   0.000      49.3123    54.68753
------------------------------------------------------------------------------------

------------------------------------------------------------------------------
  Random-effects parameters  |   Estimate   Std. err.     [95% conf. interval]
-----------------------------+------------------------------------------------
country: Independent         |
                 var(warmth) |   .0227504   .0257784      .0024689    .2096436
                  var(_cons) |   2.963975   .9737647      1.556777    5.643163
-----------------------------+------------------------------------------------
               var(Residual) |   34.97499   .9097109      33.23668    36.80422
------------------------------------------------------------------------------
LR test vs. linear model: chi2(2) = 205.74                Prob > chi2 = 0.0000

Note: LR test is conservative and provided only for reference.
\end{verbatim}

\section{Longitudinal Model}\label{longitudinal-model}

\subsection{Get Data}\label{get-data-1}

\begin{Shaded}
\begin{Highlighting}[]

\KeywordTok{use} \StringTok{"simulated\_multilevel\_longitudinal\_data.dta"}\NormalTok{, }\KeywordTok{clear}
\end{Highlighting}
\end{Shaded}

\subsection{The Equation}\label{the-equation-1}

\[\text{outcome}_{ij} = \beta_0 + \beta_1 \text{parental warmth} + \beta_2 \text{physical punishment} + \beta_3 \text{time} + \]

\[\beta_4 \text{identity}_2 + \beta_5 \text{intervention} + \beta_5 HDI +\]

\[u_{0j} + u_{1j} \times \text{parental warmth} + \]

\[v_{0i} + v_{1i} \times t + e_{ij} \]

\subsection{Descriptive Statistics}\label{descriptive-statistics-1}

\begin{Shaded}
\begin{Highlighting}[]

\KeywordTok{summarize} \CommentTok{// descriptive statistics}
\end{Highlighting}
\end{Shaded}

\begin{verbatim}
    Variable |        Obs        Mean    Std. dev.       Min        Max
-------------+---------------------------------------------------------
     country |      9,000        15.5    8.655922          1         30
         HDI |      9,000    64.76667     17.2437         33         87
      family |      9,000        50.5    28.86767          1        100
          id |          0
    identity |      9,000    .4976667    .5000223          0          1
-------------+---------------------------------------------------------
intervention |      9,000    .4843333    .4997823          0          1
           t |      9,000           2    .8165419          1          3
physical_p~t |      9,000    2.485333    1.373639          0          5
      warmth |      9,000    3.514222      1.8839          0          7
     outcome |      9,000    53.37768    6.572285   29.60798   79.02199
\end{verbatim}

\subsection{Alternate Plot}\label{alternate-plot}

\begin{Shaded}
\begin{Highlighting}[]
\KeywordTok{encode}\NormalTok{ id, }\KeywordTok{generate}\NormalTok{(idNUMERIC) }\CommentTok{// numeric version of id}
    
\NormalTok{* spagplot outcome t }\KeywordTok{if}\NormalTok{ idNUMERIC \textless{}= 10, id(idNUMERIC) }\DecValTok{scheme}\NormalTok{(stcolor)}
    
\KeywordTok{twoway}\NormalTok{ (}\KeywordTok{lfit}\NormalTok{ outcome t) (}\KeywordTok{scatter}\NormalTok{ outcome t) }\KeywordTok{if}\NormalTok{ idNUMERIC \textless{}= 10, }\KeywordTok{by}\NormalTok{(idNUMERIC) }\DecValTok{scheme}\NormalTok{(stcolor)}

\KeywordTok{graph} \KeywordTok{export}\NormalTok{ spagplot2.png, }\KeywordTok{width}\NormalTok{(1000) }\KeywordTok{replace}
\end{Highlighting}
\end{Shaded}

\begin{figure}[H]

{\centering \includegraphics[width=0.5\linewidth,height=\textheight,keepaspectratio]{spagplot2.png}

}

\caption{Alternate Plot of Outcome by Time by Individual; First 10
Observations}

\end{figure}%

\subsection{Unconditional Model}\label{unconditional-model-1}

\subsubsection{Model}\label{model-1}

\begin{Shaded}
\begin{Highlighting}[]
\NormalTok{mixed outcome || country: || id: }\CommentTok{// unconditional model}
\end{Highlighting}
\end{Shaded}

\subsubsection{ICC}\label{icc-1}

\begin{Shaded}
\begin{Highlighting}[]

\KeywordTok{estat}\NormalTok{ icc}
\end{Highlighting}
\end{Shaded}

\begin{verbatim}
Intraclass correlation

------------------------------------------------------------------------------
                       Level |        ICC   Std. err.     [95% conf. interval]
-----------------------------+------------------------------------------------
                     country |   .0748336   .0190847      .0450028    .1219141
                  id|country |   .3462837   .0171461      .3134867    .3806097
------------------------------------------------------------------------------
\end{verbatim}

\subsection{Conditional Model}\label{conditional-model-1}

\begin{Shaded}
\begin{Highlighting}[]

\NormalTok{mixed outcome t warmth physical\_punishment i.}\KeywordTok{identity}\NormalTok{ i.intervention HDI || country: warmth || id: t }\CommentTok{// multilevel model}

\KeywordTok{est} \KeywordTok{store}\NormalTok{ longitudinal }\CommentTok{// store estimates}
\end{Highlighting}
\end{Shaded}

\begin{verbatim}
Performing EM optimization ...

Performing gradient-based optimization: 
Iteration 0:  Log likelihood =  -28523.49  
Iteration 1:  Log likelihood = -28499.953  
Iteration 2:  Log likelihood = -28499.735  
Iteration 3:  Log likelihood = -28499.604  
Iteration 4:  Log likelihood = -28499.603  

Computing standard errors ...

Mixed-effects ML regression                            Number of obs =   9,000

        Grouping information
        -------------------------------------------------------------
                        |     No. of       Observations per group
         Group variable |     groups    Minimum    Average    Maximum
        ----------------+--------------------------------------------
                country |         30        300      300.0        300
                     id |      3,000          3        3.0          3
        -------------------------------------------------------------

                                                       Wald chi2(6)  = 1096.15
Log likelihood = -28499.603                            Prob > chi2   =  0.0000

------------------------------------------------------------------------------------
           outcome | Coefficient  Std. err.      z    P>|z|     [95% conf. interval]
-------------------+----------------------------------------------------------------
                 t |    .943864   .0658716    14.33   0.000      .814758     1.07297
            warmth |   .9134959   .0423732    21.56   0.000     .8304461    .9965458
physical_punishm~t |  -1.007897   .0497622   -20.25   0.000    -1.105429   -.9103647
        1.identity |  -.1276926   .1515835    -0.84   0.400    -.4247908    .1694056
    1.intervention |   .8589966   .1519094     5.65   0.000     .5612596    1.156734
               HDI |  -.0005657   .0196437    -0.03   0.977    -.0390666    .0379352
             _cons |   50.46724   1.338318    37.71   0.000     47.84418    53.09029
------------------------------------------------------------------------------------

------------------------------------------------------------------------------
  Random-effects parameters  |   Estimate   Std. err.     [95% conf. interval]
-----------------------------+------------------------------------------------
country: Independent         |
                 var(warmth) |   .0107585   .0127845      .0010477    .1104712
                  var(_cons) |   3.167087   .9146767      1.798155    5.578185
-----------------------------+------------------------------------------------
id: Independent              |
                      var(t) |   5.69e-10   1.29e-07      1.1e-202    3.0e+183
                  var(_cons) |   8.387268   .4724189      7.510624    9.366236
-----------------------------+------------------------------------------------
               var(Residual) |   26.02734   .4753703      25.11211    26.97592
------------------------------------------------------------------------------
LR test vs. linear model: chi2(4) = 1247.03               Prob > chi2 = 0.0000

Note: LR test is conservative and provided only for reference.
\end{verbatim}

\section{Nice Table of Results}\label{nice-table-of-results}

\begin{Shaded}
\begin{Highlighting}[]

\NormalTok{etable, }\KeywordTok{estimates}\NormalTok{(crosssectional longitudinal) }\CommentTok{///}
\NormalTok{showstars showstarsnote }\CommentTok{/// show stars and note}
\NormalTok{column(estimate) }\CommentTok{// column is modelname}
\end{Highlighting}
\end{Shaded}

\begin{verbatim}
                                     crosssectional longitudinal
----------------------------------------------------------------
parental warmth in past week             0.835 **      0.913 ** 
                                       (0.064)       (0.042)    
physical punishment in past week        -0.992 **     -1.008 ** 
                                       (0.080)       (0.050)    
hypothetical identity group variable    -0.300                  
                                       (0.217)                  
recieved intervention                                           
  1                                      0.640 **      0.859 ** 
                                       (0.217)       (0.152)    
Human Development Index                 -0.003        -0.001    
                                       (0.020)       (0.020)    
time                                                   0.944 ** 
                                                     (0.066)    
hypothetical identity group variable                            
  1                                                   -0.128    
                                                     (0.152)    
Intercept                               52.000 **     50.467 ** 
                                       (1.371)       (1.338)    
var(warmth)                              0.023         0.011    
                                       (0.026)       (0.013)    
var(_cons)                               2.964         3.167    
                                       (0.974)       (0.915)    
var(e)                                  34.975        26.027    
                                       (0.910)       (0.475)    
var(_cons)                                             8.387    
                                                     (0.472)    
var(t)                                                 0.000    
                                                     (0.000)    
Number of observations                    3000          9000    
----------------------------------------------------------------
** p<.01, * p<.05
\end{verbatim}

\section{QUESTIONS???}\label{majorsection}

\bookmarksetup{startatroot}

\chapter{Cross-Classified Models}\label{cross-classified-models}

\section{Introduction}\label{introduction-2}

A two level multilevel model imagines that \emph{Level 1} units are
nested in \emph{Level 2} units. A three level multilevel model imagines
that \emph{Level 1} units are nested in \emph{Level 2} units, which are
in turn nested in \emph{Level 3} units.

A cross-classified model imagines that the nesting is not hierarchical,
but rather that there are two sets of clusters or nestings which
overlap, but are not hierarchical.

\section{Get Data}\label{get-data-2}

\begin{Shaded}
\begin{Highlighting}[]

\KeywordTok{use} \StringTok{"simulated\_multilevel\_longitudinal\_data.dta"}\NormalTok{, }\KeywordTok{clear}
\end{Highlighting}
\end{Shaded}

\section{Cross Classified Model}\label{cross-classified-model}

We can treat these random effects as being \emph{cross classified}.

This might be useful if we had data where individuals lived in different
countries at different times.

However, because \texttt{id} is in fact nested inside \texttt{country},
in this case, estimating the random effects as cross classified will be
more time consuming, but will give us equivalent results to a three
level model.

\subsection{Standard (Less Computationally Efficient)
Syntax}\label{standard-less-computationally-efficient-syntax}

The below syntax will take a very long time to run with the full sample,
and thus we have commented it out.

\begin{Shaded}
\begin{Highlighting}[]
    
\NormalTok{* mixed outcome t warmth physical\_punishment || }\DataTypeTok{\_all}\NormalTok{: R.country || }\DataTypeTok{\_all}\NormalTok{: R.id}
    
\NormalTok{* }\KeywordTok{est} \KeywordTok{store}\NormalTok{ crossed1}
\end{Highlighting}
\end{Shaded}

The documentation notes that we can use a \emph{much} more
computationally efficient version of the above command, which is what we
do in these notes. The user can verify that both versions of the command
will produce equivalent results.

In fact, at the end of handout we verify the similarity of both sets of
syntax using a random sample.

\subsection{Cross Classified With Computationally Efficient
Syntax}\label{cross-classified-with-computationally-efficient-syntax}

\begin{Shaded}
\begin{Highlighting}[]

\NormalTok{mixed outcome t warmth physical\_punishment || }\DataTypeTok{\_all}\NormalTok{: R.country || id:}
    
\KeywordTok{est} \KeywordTok{store}\NormalTok{ crossed2 }\CommentTok{// store crossed effects result}
\end{Highlighting}
\end{Shaded}

\begin{verbatim}
Performing EM optimization ...

Performing gradient-based optimization: 
Iteration 0:  Log likelihood = -28516.314  
Iteration 1:  Log likelihood = -28516.277  
Iteration 2:  Log likelihood = -28516.277  

Computing standard errors ...

Mixed-effects ML regression                            Number of obs =   9,000

        Grouping information
        -------------------------------------------------------------
                        |     No. of       Observations per group
         Group variable |     groups    Minimum    Average    Maximum
        ----------------+--------------------------------------------
                   _all |          1      9,000    9,000.0      9,000
                     id |      3,000          3        3.0          3
        -------------------------------------------------------------

                                                       Wald chi2(3)  = 1168.69
Log likelihood = -28516.277                            Prob > chi2   =  0.0000

------------------------------------------------------------------------------------
           outcome | Coefficient  Std. err.      z    P>|z|     [95% conf. interval]
-------------------+----------------------------------------------------------------
                 t |   .9434605    .065866    14.32   0.000     .8143654    1.072556
            warmth |   .9053924   .0380439    23.80   0.000     .8308277    .9799572
physical_punishm~t |  -1.014385   .0499354   -20.31   0.000    -1.112257    -.916514
             _cons |    50.8301   .4123007   123.28   0.000       50.022    51.63819
------------------------------------------------------------------------------------

------------------------------------------------------------------------------
  Random-effects parameters  |   Estimate   Std. err.     [95% conf. interval]
-----------------------------+------------------------------------------------
_all: Identity               |
              var(R.country) |   3.429974    .930313      2.015668    5.836634
-----------------------------+------------------------------------------------
id: Identity                 |
                  var(_cons) |   8.608872   .4757699      7.725107     9.59374
-----------------------------+------------------------------------------------
               var(Residual) |   26.02862   .4752444      25.11363    26.97695
------------------------------------------------------------------------------
LR test vs. linear model: chi2(2) = 1260.84               Prob > chi2 = 0.0000

Note: LR test is conservative and provided only for reference.
\end{verbatim}

\section{Three Level Model}\label{three-level-model}

\begin{Shaded}
\begin{Highlighting}[]

\NormalTok{mixed outcome t warmth physical\_punishment || country: || id:  }\CommentTok{// 3 level w/ random intercepts only}
    
\KeywordTok{est} \KeywordTok{store}\NormalTok{ threelevel }\CommentTok{// store random intercept model}
\end{Highlighting}
\end{Shaded}

\begin{verbatim}
Performing EM optimization ...

Performing gradient-based optimization: 
Iteration 0:  Log likelihood = -28516.314  
Iteration 1:  Log likelihood = -28516.277  
Iteration 2:  Log likelihood = -28516.277  

Computing standard errors ...

Mixed-effects ML regression                            Number of obs =   9,000

        Grouping information
        -------------------------------------------------------------
                        |     No. of       Observations per group
         Group variable |     groups    Minimum    Average    Maximum
        ----------------+--------------------------------------------
                country |         30        300      300.0        300
                     id |      3,000          3        3.0          3
        -------------------------------------------------------------

                                                       Wald chi2(3)  = 1168.69
Log likelihood = -28516.277                            Prob > chi2   =  0.0000

------------------------------------------------------------------------------------
           outcome | Coefficient  Std. err.      z    P>|z|     [95% conf. interval]
-------------------+----------------------------------------------------------------
                 t |   .9434605    .065866    14.32   0.000     .8143654    1.072556
            warmth |   .9053924   .0380439    23.80   0.000     .8308277    .9799572
physical_punishm~t |  -1.014385   .0499354   -20.31   0.000    -1.112257    -.916514
             _cons |    50.8301   .4123007   123.28   0.000       50.022    51.63819
------------------------------------------------------------------------------------

------------------------------------------------------------------------------
  Random-effects parameters  |   Estimate   Std. err.     [95% conf. interval]
-----------------------------+------------------------------------------------
country: Identity            |
                  var(_cons) |   3.429974    .930313      2.015668    5.836634
-----------------------------+------------------------------------------------
id: Identity                 |
                  var(_cons) |   8.608872   .4757699      7.725107     9.59374
-----------------------------+------------------------------------------------
               var(Residual) |   26.02862   .4752444      25.11363    26.97695
------------------------------------------------------------------------------
LR test vs. linear model: chi2(2) = 1260.84               Prob > chi2 = 0.0000

Note: LR test is conservative and provided only for reference.
\end{verbatim}

\section{Nice Table of Results of Three Level and Cross Classified
Model}\label{nice-table-of-results-of-three-level-and-cross-classified-model}

\begin{Shaded}
\begin{Highlighting}[]

\NormalTok{etable, }\KeywordTok{estimates}\NormalTok{(threelevel crossed2), }\CommentTok{///}
\NormalTok{showstars showstarsnote }\CommentTok{/// show stars and note}
\NormalTok{column(estimate) }\CommentTok{// column is modelname}
\end{Highlighting}
\end{Shaded}

\begin{verbatim}
invalid 'showstars' 
r(198);

r(198);
\end{verbatim}

\section{Verification of Syntax Equivalence for Cross Classified
Model}\label{verification-of-syntax-equivalence-for-cross-classified-model}

\begin{Shaded}
\begin{Highlighting}[]

\KeywordTok{keep} \KeywordTok{if} \KeywordTok{family}\NormalTok{ \textless{}= 5 }\CommentTok{// random sample of families}
    
\KeywordTok{quietly}\NormalTok{ mixed outcome t warmth physical\_punishment || }\DataTypeTok{\_all}\NormalTok{: R.country || }\DataTypeTok{\_all}\NormalTok{: R.id}
    
\KeywordTok{est} \KeywordTok{store}\NormalTok{ crossed1A }\CommentTok{// less efficient syntax}
    
\KeywordTok{quietly}\NormalTok{ mixed outcome t warmth physical\_punishment || }\DataTypeTok{\_all}\NormalTok{: R.country || id:}
    
\KeywordTok{est} \KeywordTok{store}\NormalTok{ crossed2A }\CommentTok{// more efficient syntax}
    
\NormalTok{etable, }\KeywordTok{estimates}\NormalTok{(crossed1A crossed2A) }\CommentTok{///}
\NormalTok{showstars showstarsnote }\CommentTok{/// show stars and note}
\NormalTok{column(estimate) }\CommentTok{// column is modelname}
\end{Highlighting}
\end{Shaded}

\begin{verbatim}
(8,550 observations deleted)






------------------------------------------------------
                                  crossed1A  crossed2A
------------------------------------------------------
time                               0.745 **   0.745 **
                                 (0.281)    (0.281)   
parental warmth in past week       0.871 **   0.871 **
                                 (0.160)    (0.160)   
physical punishment in past week  -1.262 **  -1.262 **
                                 (0.206)    (0.206)   
Intercept                         51.755 **  51.755 **
                                 (1.009)    (1.009)   
var(R_country)                     2.245      2.245   
                                 (1.319)    (1.319)   
var(R_id)                          5.425              
                                 (1.843)              
var(e)                            23.638     23.638   
                                 (1.933)    (1.933)   
var(_cons)                                    5.425   
                                            (1.843)   
Number of observations               450        450   
------------------------------------------------------
** p<.01, * p<.05
\end{verbatim}

\section{QUESTIONS???}\label{questions}

\bookmarksetup{startatroot}

\chapter*{References}\label{references}
\addcontentsline{toc}{chapter}{References}

\markboth{References}{References}

\phantomsection\label{refs}
\begin{CSLReferences}{1}{0}
\bibitem[\citeproctext]{ref-BlandAltman1994}
Bland, J M, and D G Altman. 1994. {``Statistics Notes: Correlation,
Regression, and Repeated Data.''} \emph{BMJ} 308 (April): 896.
\url{https://doi.org/10.1136/bmj.308.6933.896}.

\bibitem[\citeproctext]{ref-DiezRoux2003}
Diez Roux, Ana. 2003. {``Potentialities and Limitations of Multilevel
Analysis in Public Health and Epidemiology.''} In \emph{Methodology and
Epistemology of Multilevel Analysis: Approaches from Different Social
Sciences}, edited by Daniel Courgeau, 93--119. Kluwer Academic
Publishers.

\bibitem[\citeproctext]{ref-FIREBAUGH20014023}
Firebaugh, Glen. 2001. {``Ecological Fallacy, Statistics Of.''} In
\emph{International Encyclopedia of the Social \& Behavioral Sciences},
edited by Neil J. Smelser and Paul B. Baltes, 4023--26. Oxford:
Pergamon.
https://doi.org/\url{https://doi.org/10.1016/B0-08-043076-7/00765-8}.

\bibitem[\citeproctext]{ref-Gelman2007}
Gelman, Andrew, Boris Shor, Joseph Bafumi, and David Park. 2007. {``Rich
State, Poor State, Red State, Blue State: What's the Matter with
{C}onnecticut?''} \emph{Quarterly Journal of Political Science} 2
(November): 345--67. \url{https://doi.org/10.2139/ssrn.1010426}.

\bibitem[\citeproctext]{ref-Nieuwenhuis2015}
Nieuwenhuis, Rense. 2015. {``Association, Aggregation, and Paradoxes: On
the Positive Correlation Between Fertility and Women's Employment.''}
\emph{Demographic Research} 32 (March).
\url{https://www.demographic-research.org/volumes/vol32/23/}.

\bibitem[\citeproctext]{ref-Oliver2015}
Oliver, Mary, and Krista Tippett. 2015. {``{M}ary {O}liver: {`{I} Got
Saved by the Beauty of the World.'}''} The On Being Project.
\url{https://onbeing.org/programs/mary-oliver-i-got-saved-by-the-beauty-of-the-world/}.

\end{CSLReferences}

\bookmarksetup{startatroot}

\chapter*{License and Citation}\label{license-and-citation}
\addcontentsline{toc}{chapter}{License and Citation}

\markboth{License and Citation}{License and Citation}

\section*{License}\label{license}
\addcontentsline{toc}{section}{License}

\markright{License}

\includegraphics[width=0.29in,height=\textheight,keepaspectratio]{88x31.png}

\emph{Multilevel Workshop} by \href{https://agrogan1.github.io/}{Andrew
Grogan-Kaylor} is licensed under a
\href{http://creativecommons.org/licenses/by/4.0/}{Creative Commons
Attribution 4.0 International License}

\section*{Citation}\label{citation}
\addcontentsline{toc}{section}{Citation}

\markright{Citation}

For attribution, please cite as:

Grogan-Kaylor (2025). \emph{Multilevel Workshop}. Retrieved from
\url{https://agrogan1.github.io/multilevel-workshop}

\section*{BibTeX Citation}\label{bibtex-citation}
\addcontentsline{toc}{section}{BibTeX Citation}

\markright{BibTeX Citation}

\begin{Shaded}
\begin{Highlighting}[]
\NormalTok{@book\{,}
\NormalTok{   author = \{Andrew Grogan{-}Kaylor\},}
\NormalTok{   city = \{Ann Arbor, MI\},}
\NormalTok{   title = \{Multilevel Workshop\},}
\NormalTok{   url = \{https://agrogan1.github.io/multilevel{-}workshop\},}
\NormalTok{   year = \{2025\},}
\NormalTok{\}}
\end{Highlighting}
\end{Shaded}





\end{document}
